\section*{Introduction}
I had the chance being assigned an internship by the IAESTE in Daejon, South Korea. It is roughly in centre of South Korea slightly on the west side. At the beginning of the internship I had already completed one year of my master studies of Computational Science and Engineering and I have been working there from September 11 to October 31. I was at the KIER, Korea Institute of Energy Research. I would rather recommend students in engineering sciences to go there as the work there is quite applied. Maybe someone with a background in applied sciences would be better suited to be sent there. A vast majority of Koreans are not able to speak English and it is worse in Daejon than in Seoul. Even among the educated Koreans English knowledge for basic conversation is lacking.

I will only write about Daejon unless another location is explicitly mentioned and focus mostly on things that surprised me. For general information about Korea and Daejon, I recommend wikitravel.
\section*{Living}
The cost distribution in Korea is very different to the one in Switzerland. To eat out and in general service is very cheap but most things in the supermarket have a similar price.  Some products like milk and European beer are even pricier. On a Sunday evening it is possible to buy a bike in a small store and offices for renting flats or houses are opened while many restaurants are closed then. 
\section*{Transportation}
In Korea the vehicles seem to have priority over the pedestrians so I strongly recommend to wait for the vehicles to pass when there is no traffic light. The fast acceleration and break is something which hard to get used to. There is only one subway line in Daejon but I never used it. Buses are in Hangeul\footnote{Korean alphabet} only so expect having to ask people when one travel a new route. Google Maps is not the main company in Korea but Naver which offers an app in Hangeul only. In Daejon Google Maps works quite well compared to Seoul but the travel time is not very reliable and I do not think that it covers all transportation.

Many Koreans drive less safe than I expected and taxi drivers are no exception. The worst are at night time. On the plus side they are very cheap.
\section*{Food}
In the institute one could have any meal for 2000 KRW\footnote{1.81 CHF} which was very cheap and saved me a lot of money. Be ready not knowing every time what one eats in a restaurant without the company of an English speaking Korean. If one is a picky eater with vegetables, seafood and fish, I would not recommend going to Korea. I enjoyed a very vast majority of Korean food. The only exception was the cafeteria which was only sometimes enjoyable and one restaurant.
\section*{Work}
I knew nothing about optical modelling and solar concentrators before my internship and I had to learn almost everything by myself. Unfortunately, my boss had changed his mind about the task that was initially meant for me. He wrote the simulation of the solar concentrator for mirrors on a central receiver himself in MATLAB. This task was the main reason why I had considered this internship and I was very disappointed. I was left to verify his code by using Soltrace, a software for general purpose optical modelling, and to finish the GUI in MATLAB. My boss was very kind and happy with my work, was very helpful for anything even not work related when available and spoke English well. Nevertheless, guessing what he meant was sometimes involved. Bosses are quite curious about one's relationship status and are not afraid to ask.
\section*{Being a foreigner}
I am very tall (1m92) and I was told frequently that I am tall and this happened even when just walking around. I was not very tall any more, I was a giant. Everything there is just too small and too low. 
In general, people to stare at foreigners outside touristic places. As a foreigner one is special and one gets a special treatment. I was almost denied with other foreigners to enter a bar/club (HO bar) and there was a club where foreigners had free entrance but Koreans not. Furthermore, I got spontaneously a free ride in Busan on a jet ski or additional free food in a restaurant twice.
\section*{Local committee}
The local committee where for me very nice shepherds. When I didn't know how to get home or if I needed any advice they were always there and helped very fast. I want to say a big thank you to Cayte, Choonmee, Wooyun and Suah for taking care so well of me and for their generosity.

There were two big events during my stay. One was the visit of the demiliratised zone, the Korean barbecue in a guest house and the river rafting for the IAESTE members only. The second one was the K-Fan Field Trip with the IAESTE trainees and the foreigners working in the KIER. There we visited the POSCO steelworks and the KOSPO thermal plant on the first day and then spent the night in the Gongju Hanok village. On the second day, we went to the National Museum of Korean Contemporary History and a fashion event/exposition. Actually, the best part was the Korean restaurants. I had the best Korean food of all my stay then and everything was sponsored by the KIER! The trips were really nice and would have been hard to organize without help of locals.
\section*{People}
Thanks to the people staying there, especially my fellow interns with whom I spent most of my free time, my stay in Korea was a great one. I had a nice roommate from India who invited me to eat with his fellow Indians staying in the dormitory what they had prepared what they are used from their home country. At the KIER, there were mostly Koreans and the foreigners where mostly Master's/Ph.D. students from Pakistan and Indonesia and postdoctoral scholars from India. It was great to meet people from a very different culture while dealing with another different culture and being able to find different common denominators. 
\section*{Conclusion}
My IAESTE internship was a very enriching experience. A stay abroad is something quite different to travelling and every weekend is like holidays because everything is so new and fascinating and one can just live and enjoy the moment instead of worrying about the future. It is nice to step back from Switzerland to see things from a long distance and to leave one's comfort zone to explore.
\appendix
\onecolumn
\section*{Pictures}
\begin{figure}[h]
	\begin{center}
		\includegraphics*[width=\columnwidth]{photo1.jpg}
	\end{center}
	\caption{A nice view in Busan of the sea.}
	\label{fig:photo1}
\end{figure}
\begin{figure}[]
	\begin{center}
		\includegraphics*[width=\columnwidth]{photo3.jpg}
	\end{center}
	\caption{Me experiencing sightseeing in Busan with all my senses.}
	\label{fig:photo1}
\end{figure}
\begin{figure}[]
	\begin{center}
		\includegraphics*[width=\columnwidth]{photo2.jpg}
	\end{center}
	\caption{The forest in or near Daejon.}
	\label{fig:photo1}
\end{figure}
\begin{figure}[]
	\begin{center}
		\includegraphics*[width=\columnwidth]{photo4.jpg}
	\end{center}
	\caption{The forest in or near Daejon again. I was even wearing the IAESTE shirt on that day!}
	\label{fig:photo1}
\end{figure}
