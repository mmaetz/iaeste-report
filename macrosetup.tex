%% Custom commands =============================================================
\newlength\dlf
\newcommand\alignedbox[2]{
  % #1 = before alignment
  % #2 = after alignment
	\hphantom{#1\,}
  &
  \begingroup
  \settowidth\dlf{$\displaystyle #1$}
  \addtolength\dlf{\fboxsep+\fboxrule}
  \hspace{-\dlf}
  \boxed{#1 #2}
  \endgroup
}
\newcounter{subsectionslide}
\newcommand{\ifcontent}[1]{\ifthenelse{\equal{#1}{}}{}{\ifthenelse{\equal{\arabic{subsectionslide}}{0}}{\stepcounter{section}\addcontentsline{toc}{section}{\protect\numberline{\thesection}#1}}{\stepcounter{subsection}\addcontentsline{toc}{subsection}{\protect\numberline{\thesubsection}#1}}}}
\newcommand{\secsl}{\setcounter{subsectionslide}{0}}
\newcommand{\subsecsl}{\stepcounter{subsectionslide}}

\newcounter{lecturenumber}
\newcounter{slidenumber}[lecturenumber]
\newcounter{sketchnumber}
\newcommand{\slidewidth}{1\textwidth-2pt}
\newcommand{\lecturename}{}
%\newcommand{\psd}[7]{\begin{center}%
%	\ifcontent{#5}
%	\includegraphics[page=#2,width=#4\textwidth]{../mod-slides/#1.pdf}\\%
%	\ifcontent{#6}
%	\includegraphics[page=#3,width=#4\textwidth]{../mod-slides/#1.pdf}%
%\end{center}#7}
\newcommand{\psl}[2]{
	\ifthenelse{\equal{\lecturename}{}}{
\stepcounter{slidenumber}
\begin{center}
		\begin{tikzpicture}
			\node[rectangle,draw,ultra thick,minimum width=\slidewidth,minimum height=0.5625\slidewidth] (a) at (0,0) {\LARGE Slide \#\arabic{slidenumber}};
		\end{tikzpicture}\\
\end{center}#2
	}{
	\begin{center}%
\stepcounter{slidenumber}
	\ifcontent{#1}
	{%
	\setlength{\fboxsep}{0pt}%
	\setlength{\fboxrule}{1pt}%
	\fbox{\includegraphics[page=\arabic{slidenumber},width=\slidewidth]{../slides/\lecturename.pdf}}%
	}%
\end{center}#2
}}
\newcommand{\psk}{
	\begin{center}%
		\stepcounter{sketchnumber}
		\setlength{\fboxsep}{0pt}%
		\setlength{\fboxrule}{1pt}%
		\fbox{\includegraphics[page=\arabic{sketchnumber},height=\slidewidth,angle=90]{sketches/numpde-sketches.pdf}}%
		\includepdf[pages=\arabic{sketchnumber}]{sketches/numpde-sketches.pdf}
	\end{center}
}
\newcommand{\Include}[2]{\renewcommand{\lecturename}{#1}\stepcounter{lecturenumber}\mbox{}\\
\begin{minipage}[]{\textwidth}
	\medskip\hrule \smallskip\hfill lecture \#\arabic{lecturenumber}\smallskip\hrule\medskip
\end{minipage}
\\\input{lectures/#2}}

%% Depends on this
\usepackage{ifthen}
\usepackage{rotating}
%\usepackage[all]{xy}

%% Differential d. This sets it as proper operator in roman type. With correct
%% spacing. ISO standards for mathematical typesetting says it should be printed
%% like this.
%\newcommand{\diff}[1]{\operatorname{d}\ifthenelse{\equal{#1}{}}{\,}{\!#1}}
\newcommand{\dd}[1]{\operatorname{d}\ifthenelse{\equal{#1}{}}{\,}{\!#1}}
\newcommand{\rd}{\mathrm d}

%% Symbols for euler number and imaginary unit
\providecommand*{\eu}%
{\ensuremath{\mathrm{e}}}
% The imaginary unit
\providecommand*{\iu}%
{\ensuremath{\mathrm{i}}} % i can be replaced with j on preference.

%% Uncomment below what style you prefer for printing differential operators
%\DeclareMathOperator{\grad}{grad}
%\DeclareMathOperator{\rot}{rot}
%\DeclareMathOperator{\Div}{div}
\DeclareMathOperator{\et}{ET}
\DeclareMathOperator{\grad}{\nabla}
\DeclareMathOperator{\rot}{\nabla\times}
\DeclareMathOperator{\Div}{\nabla\cdot}

%% Additional mathematical operators
%% Just use the physics package
\DeclareMathOperator{\spn}{span}
\DeclareMathOperator{\diam}{diam}
\DeclareMathOperator{\supp}{supp}
\DeclareMathOperator{\sgn}{sgn}
\DeclareMathOperator{\tr}{tr}
\DeclareMathOperator{\id}{Id}
\DeclareMathOperator{\arccot}{arccot}
\DeclareMathOperator{\arsinh}{arsinh}
\DeclareMathOperator{\arcosh}{arcosh}
\DeclareMathOperator{\artanh}{artanh}
%% German variants
%\DeclareMathOperator{\Kern}{Kern}
%\DeclareMathOperator{\Bild}{Bild}
%\DeclareMathOperator{\Grad}{Grad}
%% English variants
%% \ker is provided by LaTeX
\DeclareMathOperator{\im}{im}
%% \grad is provided by LaTeX

%% Special characters for number sets, e.g. real or complex numbers.
\newcommand{\C}{\mathbb{C}}
\newcommand{\K}{\mathbb{K}}
\newcommand{\N}{\mathbb{N}}
\newcommand{\Q}{\mathbb{Q}}
\newcommand{\R}{\mathbb{R}}
\newcommand{\Z}{\mathbb{Z}}
\newcommand{\X}{\mathbb{X}}

%% Fixed size delimiter examples
\newcommand{\floor}[1]{\lfloor #1 \rfloor}
\newcommand{\ceil}[1]{\lceil #1 \rceil}
\newcommand{\seq}[1]{\langle #1 \rangle}
\newcommand{\set}[1]{\{ #1 \}}
\newcommand{\abs}[1]{\left\lvert #1 \right\rvert}
\newcommand{\norm}[1]{\left\lVert #1 \right\rVert}
\newcommand{\hzonorm}[1]{\left\lVert #1 \right\rVert_{H_{0}^{1}}}
\newcommand{\honorm}[1]{\left\lVert #1 \right\rVert_{H_{1}}}
\newcommand{\htnorm}[1]{\left\lVert #1 \right\rVert_{H_{2}}}
\newcommand{\htonorm}[1]{\left\lVert #1 \right\rVert_{H_{2}(\Omega)}}
\newcommand{\vnorm}[1]{\left\lVert #1 \right\rVert_{V}}
\newcommand{\ltnorm}[1]{\left\lVert #1 \right\rVert_{L^{2}}}
  
\newcommand{\ssp}[2]{\left\langle #1,#2 \right\rangle}
\newcommand{\isp}[2]{\left(  #1,#2 \right) }
\newcommand{\hzosp}[2]{\left(  #1,#2 \right)_{H_0^1} }
\newcommand{\hosp}[2]{\left(  #1,#2 \right)_{H_1} }
\newcommand{\htsp}[2]{\left(  #1,#2 \right)_{H_2} }
\newcommand{\lsp}[2]{\left(  #1,#2 \right)_{L^2} }
\newcommand{\indic}[1]{\bigl[#1\bigr]}
\newcommand{\comm}[2]{\left[ {#1}, {#2} \right] }
\newcommand{\acomm}[2]{\left\{ {#1}, {#2} \right\} }

%% Scaling delimiter examples
\newcommand{\Floor}[1]{\left\lfloor #1 \right\rfloor}
\newcommand{\Ceil}[1]{\left\lceil #1 \right\rceil}
\newcommand{\Seq}[1]{\left\langle #1 \right\rangle}
\newcommand{\Set}[1]{\left\{ #1 \right\}}
\newcommand{\Abs}[1]{\left\lvert #1 \right\rvert}
\newcommand{\Norm}[1]{\left\lVert #1 \right\rVert}

%% Absolute and partial derrivate fractions
%% Calculus
\newcommand{\dv}[2]{\frac{\mathrm d #1}{\mathrm d #2}}
\newcommand{\dvt}[3]{\frac{\mathrm d ^{2} #1}{\mathrm d #2 ^{2}}}
\newcommand{\dvn}[3]{\frac{\mathrm d ^{#1} #2}{\mathrm d #3 ^{#1}}}
\newcommand{\pdv}[2]{\frac{\partial #1}{\partial #2}}
\newcommand{\pddv}[2]{\frac{\partial^{2} #1}{\partial #2 ^{2}}}
\newcommand{\pddvm}[3]{\frac{\partial^{2} #1}{\partial #2 \partial #3}}
\newcommand{\pdvn}[3]{\frac{\partial^{#1} #2}{\partial #3 ^{#1}}}
\newcommand{\fdv}[2]{\frac{\delta #1}{\delta #2}}
\newcommand{\vtr}[1]{\boldsymbol{\mathrm{#1}}}


%% Set an index and print it to the current position at the same time
\newcommand{\Index}[1]{\emph{#1}\index{#1}}

%% Displaystyle math for inline math mode
\newcommand{\ds}{\displaystyle}

%% Easy to use alias for the default matrices with round braces
\newcommand{\Mx}[1]{\ensuremath{\begin{pmatrix}#1\end{pmatrix}}}

%% Include a lecture from the lectures/ folder by date.
%% Added ddmmyydate option because the default format is too large.


%% A macro to typeset a commutitive diagram in the style of
%% \[\Abb[functionname]{from}{to}{fromelement}{toelement}\]
\newcommand{\Sidein}{\begin{rotate}{90}\small$\in$\end{rotate}}
\newcommand{\sidew}[1]{\rotatebox{90}{\small$#1$}}

\newcommand{\Abb}[5][]{\ensuremath{
    \begin{array}{lc}
      \ifthenelse{\equal{#1}{}}{}{#1:}\;\; &
      \begin{xy}
        \xymatrixrowsep{1em}\xymatrixcolsep{2em}%
        \xymatrix{ #2 \ar[r] \ar@{}[d]^<<<<{\hspace{0.001em} \Sidein}
          & #3  \ar@{}[d]^<<<<{\hspace{0.001em} \Sidein} \\
          #4 \ar@{|->}[r] & #5} \end{xy}
    \end{array}
  }%
}


%% Use the alternative epsilon per default and define the old one as \oldepsilon
\let\oldepsilon\epsilon

\renewcommand{\epsilon}{\ensuremath\varepsilon}

%% Also set the alternate phi as default.
%\renewcommand{\phi}{\ensuremath{\varphi}}

%% Uncomment to type bra, kets etc.
\newcommand{\bra}[1]{\left\langle #1\right|}
\newcommand{\ket}[1]{\left| #1\right\rangle}
\newcommand{\braket}[2]{\left\langle #1\right|\left.\! #2\right\rangle}
\newcommand{\ketbra}[2]{\ensuremath{ {\ket{#1} \!\bra{#2}}}}
\newcommand{\proj}[1]{\ensuremath{ {\ket{#1} \!\bra{#1}}}}
\newcommand{\sumproj}[1]{\ensuremath{ {\sum_{#1}\proj{#1}}}}
%% matrix element
\newcommand{\mel}[3]{\ensuremath{\left\langle {#1}\vphantom{#3} \right\rvert{#2} \left\lvert{#3}\vphantom{#1}\right\rangle}}
\newcommand{\ev}[3]{\ensuremath{\left\langle {#1}\right\rvert{#2} \left\lvert{#1}\right\rangle}}

\newcommand{\vevj}[2]{\left\langle {#1} \right\langle_{#2} }
\newcommand{\vev}[1]{\left\langle {#1} \right\rangle }

\newcommand{\cmark}{{\color{black!30!green}\ding{51}}}
\newcommand{\xmark}{{\color{black!30!red}\ding{55}}}
